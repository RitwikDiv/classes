\documentclass[]{article}
\usepackage[margin=2cm]{geometry}
%opening
\title{Spring 2015 PURE Application}
\author{Adam Sumner}
\date{\textbf{Due:} January 23rd, 2015 at 5:00pm}
\begin{document}

\maketitle


\section{Research Description and Relevance}
The goal of "Faster than Real-Time Power System Dynamic Simulation" is to provide actionable information to operators during power system disturbances and cascading outages. The project is developing high performance computer modeling and simulation software for predicting the effects of disturbances faster than real-time. Based on the predictive capabilities of this research, operators will be able to respond before the full effects of a cascading outage are realized, thereby avoiding wide spread blackouts.
\subsection{Topic Area}
Power System Dynamics Simulation
\subsection{Research Questions}
\begin{enumerate}
	\item Are current power system models capable of predicting the full effects of an initiating disturbance? If not, what is missing?
	\item Are current power system simulation techniques capable of predicting the full effects of an initiating disturbance fast enough to aid an operator in the control room? If not, what are the bottlenecks?
	\item What role, if any, does dynamic load (e.g. induction motor) play in a cascading outage?
	\item What role, if any, does the generation control system play in a cascading outage?
	\item What role, if any, does the grid control system play in a cascading outage?
	\item What role, if any, does the relaying \& protection system play in a cascading outage?
\end{enumerate}
\subsection{Research Method}
\begin{enumerate}
	\item The team will implement various industry standard control device models to see what role they play in a cascading outage.
	\item During the implementation process, the team will explore various data structures to increase the program's reliability and performance.
	\item The control device model responses will be benchmarked in a small test system against a commercial software tool for offline power system dynamics simulation.  
\end{enumerate}
\subsection{Relevance}
The project will accelerate performance and enhance accuracy of dynamics simulations, enabling operators to maintain reliability and steer clear of blackouts. This effort will form the backbone of a new hybrid real-time protection and control architecture, which will increase the reliability and efficiency of our electric energy delivery systems.
\subsection{Individual Role}
Adam has previously worked on this project before, implementing several exciter models into the simulation program. He is knowledgeable in how to successfully implement the various industry standard control device models, how to debug the code, and how to benchmark results against PSSE for verification. During the course of this semester, he will be tackling a new aspect of the program. This will involve  working on implementing the data import routines necessary to read CYME distribution system data. The existing simulation program has been developed for transmission grids, but the CYME import capability will enable the group to analyze distribution systems and eventually transmission and distribution systems simultaneously.

\section{Student Qualifications}
As previously stated, Adam has had previous experience working on this project. He is knowledgeable of what the project entails, and is a quick learner. He is currently in his third year at IIT, and has many goals and aspirations, one being the possibility of attending graduate school to conduct research of his own. Recently he has been named an IEEE PES Scholar, which is a scholarship for students interested in a career in the power and energy field that was given to 184 students across the United States and Canada. Furthermore, Adam has previously worked as an intern at ComEd, allowing him to see how his classroom studies apply to real world applications. He has also received the Howard P. Allen scholarship from Midwest Generation, and the Grainger Power Engineering scholarship. Not only has he received several scholarships relating to the field of power engineering, but has also received several other academic awards, one being a Camras Scholar here at IIT. Adam is very involved on campus and is an ARC tutor in both areas of ECE and CS, a member of Phi Kappa Sigma, Tau Beta Pi, Eta Kappa Nu, and IEEE. He also is involved with the Vandercook School of Music, playing drumset in the New Orleans Jazz Combo. Even though he is involved with many activities, he is an expert at balancing his academics with the work he does outside of school. He has so far maintained a college Cumulative GPA of 4.0 and continues to keep academic excellence his topmost priority. In relation to the research project, Adam has had formal training and practice in the C programming language, which is heavily used in this project. Furthermore, he loves mathematics and has the necessary problem solving skills to tackle whatever is presented to him. He is a devoted individual and puts everything he can into the work that he does. 
\section{Faculty Mentoring Plan}
The undergraduate research group will meet once a week. The team will work together in the power engineering computing laboratory. Each student will enter his/her work assignments and contributions into a daily journal kept online. In addition, an "issue" log will be created for tracking all problems and potential solutions. Professor Flueck will review the journal and the issue log throughout the week to provide suggestions to the research group. Simulation data, project-related tutorials, and other documentation will be hosted on Professor Flueck's collaboration and project management server. Source code will be hosted on the git server currently used by the MS and PhD students. Each member of the undergraduate research group will contribute code to the modeling and simulation platform for power system dynamics. In addition, each student will be involved in the preparation of project reports and poster presentations for internal and external audiences.
\end{document}
