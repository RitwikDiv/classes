\documentclass[12pt]{article}
\usepackage[margin=3cm]{geometry}
\usepackage{url}
%opening
\title{\LARGE\textsc{Topic 1: Analyzing the Competitive Space for Psychology Tools, a Case Study of Sona Systems}}
\author{Adam Sumner}
\date{\textbf{Due: }May 6\textsuperscript{th}, 2016 at 11:59pm}

\begin{document}
\maketitle
\section{Introduction}
Psychologists at many universities use a myriad of tools in their daily lives. Without these tools, it would not be possible for them to easily complete their daily tasks/goals. Thus, an industry of software has been created to facilitate this need. Popular amongst psychology departments across the United States is a research tool called Sona. In fact, Sona is in use in six out of seven continents on planet Earth. This tool is a necessity to both faculty and researchers. Without Sona, many psychologists at renown universities would be crippled. However, there has not been a popular competitor in the marketplace to rival Sona. This leaves many universities with no option other than to purchase Sona, and with a lack of competitors, Sona has had no need to improve its features. This has brought major problems to universities, leaving them with no choice other than having to deal with Sona's decisions in their product design. While Sona can be considered a great tool, there is still room for other products to enter the monopolized market that it has conquered. 

\section{Sona}
Sona Systems is a cloud-based participant pool management solution that helps administrators and researchers launch full-scale research, provides quality user-experience, and helps access participant information or study schedules anywhere in the world. It does this through an array of web-accessible devices. It eliminates the need for traditional paper methods by virtually integrating every function of the research administration process online. From the participant sign-up process and prescreening, to the automating of email communication and real-time research reporting, Sona Systems provides a solid foundation for enhancing student engagement\cite{sona}. This is what Sona Systems advertises as their overview of their product. While this description may be accurate in its overview description, it lacks in thoroughly describing its product in depth.
\subsection{Cost and Features}
The main features of Sona include:
\begin{itemize}
	\item Setting up Studies
	\item Prescreening Participants
	\item Generating Reports
	\item Managing Schedules
	\item Creating Surveys
	\item Third Party Integration
\end{itemize}
 Sona is a subscription based software that will generally cost universities roughly \$1200 a year. With this subscription, universities gain an administrator account and unlimited control over how many participant accounts may be created, along with how many research studies may be published. While it does offer its own embedded survey implementation, researchers generally rely on Qualtrics, a 3\textsuperscript{rd} party research publishing tool, to create their surveys. Furthermore, Sona Systems is limited to the university and its students. What this means is that if for example a university like IIT were to subscribe, only IIT students are able to participate in the studies posted. This also only allows researchers at IIT to post their studies.
\subsubsection{Credit Management}
One of the main reasons Psychology departments utilize Sona is for its credit management capabilities. Many freshman level psychology classes require students taking these courses to participate in campus research studies. In fact, participation generally factors in to roughly five to ten percent of a student's final grade in the class. With its ability to host studies at a central online hub, Sona also comes packaged with the capability to track how many studies each user participates in and completes. Professors of each course can then track each students progress in their class towards whatever goal they set for their own respective course. 
\section{Shortcoming/Opportunities}
While Sona at its surfaces provides an amazing set of features for its users, its execution and application of these features are where users feel they could be improved. For example, Sona offers prescreening for its candidates for all of the studies posted. While this is a great feature, in practice, it requires users to answer many of the same answers for each study they attempt to participate in.
\begin{thebibliography}{9}
	\bibitem{sona}
	\url{https://www.sona-systems.com/overview.aspx}
\end{thebibliography}
\end{document}
