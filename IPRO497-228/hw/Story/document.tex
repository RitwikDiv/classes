\documentclass[12pt]{article}
\usepackage[margin=3cm]{geometry}
\usepackage{indentfirst}
%opening
\title{\textbf{Prototype Story}}
\author{IPRO 497-228 \vspace{1cm}\\ Adam Sumner\\ Apisit Areewong\\Jusuf Skelic\\ Sang-il Yim }
\date{}

\begin{document}

\maketitle

\section*{Life Before the Birth of Our Prototype}
Our story begins in the south side of Chicago, an oasis filled with intellectuals located between the neighborhoods of Bridgeport and Bronzeville, the renown university Illinois Institute of Technology. Many disciplines are taught here: Engineering, Design, Law, Computer Science, etc., but our story focuses on the lovely psych department, a subsect of The Lewis College of Human Sciences. Long ago the bureaucracy chose to implement a mandatory rule that students participating in introductory psych courses must participate in research studies, but they didn't know how to adequately monitor them. With a decision made, and no immediate solution to their problem, an executive decision was made to outsource this issue, thus introducing our good friend Sona to our story.

Sona is a mystical thing, no not a person, but a well constructed piece of software developed by the programmers over at Sona Systems Ltd. It is a survey management tool that allows users to schedule in person appointments, take online surveys, and allot credit to those who complete said surveys. With this functionality, it seemed almost outright preposterous for the psych department to try and develop an in house equivalent system. Surely this would solve their problem. But alas, there was a catch. See, while Sona offered promise for its customers, it used the wrong business approach to making itself accessible to them. This approach is of course a subscription service. While many products like magazines, massively multiplayer online games, and online media streaming services use this method, a product like Sona doesn't really fit in. These example products require upkeep like utilizing servers for streaming/allowing players to connect to an in game server or consistently producing new material for each month of the year. Sona however, offers one thing, which need not any sort of upkeep nor drastic monthly changes. Nonetheless, seeing no other option, the psych department caved and subscribed for a quick fix. Little did they know that in the future a savior group would appear.

\section*{Rise of the Challenger}
Now an important detail I previously left out of this story was a requirement of all undergraduate students hoping to obtain their four year degree. This requirement is that each student must participate in two Interprofessional Projects. These projects require students of multiple disciplines to team up and use their individually obtained skills from their studies to develop innovative solutions to cutting edge problems of the current year. With the introduction of this new program, the psych department knew they had an out to the evil clutches of Sona. Arlen C. Moller, the current admin of IIT's Sona software quickly contacted those in charge of the program, and the experienced design experts Twisha Shah-Brandenburg and Carrie Blumenfeld were assigned with the task of recruiting a stellar team of bright young adults willing to take on the challenge. Through the use of myIIT, students quickly raced to secure their spot on the team. In the first week 14 students arrived, but as this quest was not easy, only 11 survived until the next week. Once the worthy students had proven themselves, they were divided into three teams, each in charge of developing a supreme version of Sona. The games had begun.

\section*{The Struggle for Supremacy}
One team, comprised of the outstanding and accomplished students Adam Sumner, Apisit Areewong, Jusuf Skelic, and Sang-il Yim knew they would reign victorious in the end. With their focus on appeasing the faculty, the people who are directly affected by Sona's behind the scenes features, they knew they could design a better cost efficient system. With guidance from their two wise mentors, a prototype was developed. This prototype had it all: an easy login system utilizing BlackBoard and myIIT integration, an intuitive layout for its users, grade synchronization, and most importantly, the opportunity for expansion. Yes, this system would not only be developed for a specific university, but it would be designed so that any university would be able to use this system. This was the beginning to the end of the Sona era of higher education. With approval from Arlen, Twisha, and Carrie, and with minor adjustments to the initial phase of Sona's challenger software system, the psych department's liberator had been born.
\newpage
\section*{The Current Situation}
With a great premise and talented team members, the prototype is on its way. Yes it will have to fend off two other challengers to make it to the final stage, but the team knows their system will prevail. Only time will tell if their premonition will come true...
\end{document}
