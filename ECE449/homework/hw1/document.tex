\documentclass[12pt]{article}

%opening
\title{\textbf{Homework 1}}
\author{Adam Sumner \\ ECE 449}
\date{October 16\textsuperscript{th}, 2015}
\begin{document}

\maketitle

\begin{abstract}
This paper addresses the concepts of computer simulation in relation to large scale systems, in particular, those concerning big data, sensor networks, and the Internet of things. It will first introduce what computer simulation is and its benefits and then discuss its applications in the modern world.
\end{abstract}

\section{Computer Simulation}
Computer simulation is an important tool that both scientists and engineers use in industry. It allows system designers to build their design, test its functionality, and change features through the use of software. This allows both companies and designers to become extremely cost efficient due to the fact that they can virtually create systems that are essentially cost free. Furthermore, a designer can quickly implement or add modular functionality to an already designed system, thus also promoting extensibility in their design. It is an extremely useful tool, which is why it's necessary for engineers to harness its power.

\section{Applications}
Computer simulation is used in a plethora of fields. In particular, when designing large scale transmission and distribution systems, simulation is a must. Due to not only the complexity of these systems, but the infrastructure and cost, computer simulation is the only way to design a proper power grid. There are plenty of buses, capacitor banks, inductors, transformers, distribution automation devices, and three separate phases of power to worry about. In particular, a program called PSSE, developed by Siemens, is widely used at many electric utility companies, including ComEd which supplies the Chicago area with power. Furthermore, there is a new piece of simulation software being developed at IIT by Dr. Alexander J. Flueck in conjunction with Argonne National Lab called TS3ph which tries to address real time and future prediction of faults on the power grid. Upon completion of this project, electric utility companies will be able to predict where faults will occur, thus preventing cascading outages, and increasing response times for operators and linemen to keep customers connected to a source of power.

Computer Simulation is incredibly useful in the field of Very Large Scale Integration(VLSI) design as well. VLSI design deals with placing billions of transistors on an integrated chip to construct things such as Arithmetic Logic Units, Central Processing Units, and Graphics Processor Units. Once constructed, these devices are able to create an Internet of things which can be used to create embedded systems, or even a personal computer so that the user can go on their favorite website to look at funny pictures of cats. It would be a huge waste of money and time to design a theoretical system that involves the interrelation of millions to billions of transistors and push the design into production only to find out that there are plenty of bugs that could've been fixed if caught earlier. This is why VLSI designers use programs such as HSPICE to test their designs, verify functionality, and create more power efficient systems to combat the current ``Power Wall" issue in many processors and modern smart phones. If computer simulation did not exist, technology would not be as advanced as it is now, and companies would be wasting money by manufacturing terrible prototypes. The big takeaway is that testing devices and large scale systems is incredibly time consuming. With computer simulation, one person can simulate a large system with a simple Unix command.

\section{Conclusion}
Overall computer simulation is one of the most important tools an engineer can use in his/her daily activities. It not only promotes efficiency, but cost effectiveness as well. Simulation is the key to future growth in systems design.

\end{document}
