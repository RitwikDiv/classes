\documentclass[12pt]{article}
\usepackage[margin=3cm]{geometry}
\usepackage{graphicx}
\usepackage{float}
\usepackage{url}
\usepackage{listings}
\usepackage{color}

\definecolor{mygreen}{rgb}{0,0.6,0}
\definecolor{mygray}{rgb}{0.5,0.5,0.5}
\definecolor{mymauve}{rgb}{0.58,0,0.82}
\lstset{ %
	xleftmargin=10em,
	backgroundcolor=\color{white},   % choose the background color; you must add \usepackage{color} or \usepackage{xcolor}
	basicstyle=\small,%\footnotesize,        % the size of the fonts that are used for the code
	breakatwhitespace=false,         % sets if automatic breaks should only happen at whitespace
	breaklines=false,                 % sets automatic line breaking
	captionpos=b,                    % sets the caption-position to bottom
	commentstyle=\color{mygreen},    % comment style
	deletekeywords={...},            % if you want to delete keywords from the given language
	escapeinside={\%*}{*)},          % if you want to add LaTeX within your code
	extendedchars=true,              % lets you use non-ASCII characters; for 8-bits encodings only, does not work with UTF-8
	%	frame=single,                    % adds a frame around the code
	keepspaces=true,                 % keeps spaces in text, useful for keeping indentation of code (possibly needs columns=flexible)
	keywordstyle=\color{blue},       % keyword style
	language=C, % the language of the code
	morekeywords={*,...},            % if you want to add more keywords to the set
	numbers=left,                    % where to put the line-numbers; possible values are (none, left, right)
	numbersep=5pt,                   % how far the line-numbers are from the code
	numberstyle=\small\color{mygray}, % the style that is used for the line-numbers
	rulecolor=\color{black},         % if not set, the frame-color may be changed on line-breaks within not-black text (e.g. comments (green here))
	showspaces=false,                % show spaces everywhere adding particular underscores; it overrides 'showstringspaces'
	showstringspaces=false,          % underline spaces within strings only
	showtabs=false,                  % show tabs within strings adding particular underscores
	stepnumber=1,                    % step between two line-numbers. If it's 1, each line will be numbered
	stringstyle=\color{mymauve},     % string literal style
	tabsize=2                  % sets default tabsize to 2 spac                  % show the filename of files included with \lstinputlisting; also try caption instead of title
}

\begin{document}

\begin{titlepage}
	\begin{center}
		
		
		% Upper part of the page. The '~' is needed because \\
		% only works if a paragraph has started.
		\vfill
		
		\textsc{\LARGE Lab 9: Standard Cell Based ASIC Design Flow}\\[1.5cm]
		
		\Large Adam Sumner\\[0.5cm]
		
		\Large Illinois Institute of Technology\\[0.5cm]
		
		\Large ECE 429-01\\[0.5cm]	
		
		\noindent
		\vfill
		\large \textbf{Lab Date:} November 9\textsuperscript{th}, 2015\hfill
		\large \textbf{Due Date:} November 16\textsuperscript{th}, 2015
		% Bottom of the page
	
		
	\end{center}
\end{titlepage}

\section{Introduction}

\section{Theory/Pre-Lab}
\subsection{Theory}
Standard cell based ASIC design flow automatically synthesizes a chip layout from a register transfer level (RTL) description of a chip. The design flow utilizes the standard cell library to synthesize a chip layout according to design constraints including cost, performance, power consumption, etc. While this may seem like a nice tool to completely replace manual chip design, in practice, the tool will seldom generate a satisfactory design in the beginning. It is necessary for chip designers to understand the various steps of the design flow to guide the tool through multiple iterations before an optimum solution is found. In general, the standard cell based ASIC design flow consists of two steps: Logic Synthesis and Physical Design.
\subsubsection{Standard Cell Library}
A standard cell is a logic gate with a lay



\subsection{Pre-Lab}

 






\section{Implementation}
\subsection{Schematics}





\subsection{Procedure}


\subsection{Results}

\subsection{Discussion}




%\subsection{Bonus Work}
\section{Conclusions}


\end{document}
