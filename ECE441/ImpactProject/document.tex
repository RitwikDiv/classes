\documentclass[12pt,letter]{article}
\usepackage{url}

%opening
\title{\textbf{Technology Impact Project}}
\author{Adam Sumner}
\date{April 20th, 2015}

\begin{document}

\maketitle

\begin{abstract}
This paper addresses the impact that IIT Alumni Marvin Camras and Martin Cooper have had on the technology industry. The major achievements of each will be recognized, and their impact on society will be compared to each other. The social, economic, ethical, and environmental impacts of their innovations will be addressed, as well as how their innovations affect current and future technological trends. Last, concerns on rapid technology changes will be discussed.
\end{abstract}

\section{Introduction}
IIT and its alumni have had a great impact on society, especially in the world of technological advances. Two of the most notable alumni from the ECE department are Marvin Camras and Martin Cooper. Through their innovations, they paved the way for modern day technology that we use everyday. Camras paved the way for technological advances in the field of magnetic recording, while Cooper revolutionized the wireless communication industry. It is because of them that we are able to use seemingly simple technology such as smart phones and think nothing of it.

\section{Discussion and Answers to Questions}
\subsection{The Major Achievements of Marvin Camras}
Mavin Camras was an electrical engineer and an inventor who was a major influence in the field of magnetic recording. His first device that he built was a simple recording device. This wire recorder was a simple type of storage device that used steel/stainless steel wire to store analog audio\cite{wire}. After this he discovered that using magnetic tape makes the process of splicing audio samples much easier. In June of 1944, he was awarded his first patent, ``US Patent Number 2351004", and later he was awarded over 500 more \cite{patent}. On top of this, Camras wrote a predictive paper titled ``Magnetic Recording and Reproduction - 2012 A.D" which accurately predicted the existence of portable media players the size of a pack of cards without any mechanically moving parts. It is clear now, that his predictions were correct.
\subsection{The Major Achievements of Martin Cooper}
Martin Cooper, like Marvin Camras, was also an electrical engineer and inventor. While working at Motorola, he developed many critical technological advancements that are still in use today, although much more advanced. In 1967, he created the first cellular-like portable handheld police radio system\cite{coop}. In the early 70's, Cooper was head of Motorola's communications division, where he started to work on his idea of bringing portable cellular phones to the public market. A technology at that time that allowed people to communicate ``portably" was called a car phone, however, they were in limited use. Furthermore, they were specifically designed for use in an automobile, which placed restrictions on its use.
\subsection{Global Technological Impacts}
\subsection{Social, Economical, Ethical, and Environmental Impacts}
\subsection{Future Technological Trends}
\subsection{Rapid Technological Changes}
\section{Conclusion}

\bibliographystyle{plain}
\bibliography{master}
\end{document}
