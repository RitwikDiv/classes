\documentclass[12pt,letter]{article}
\usepackage{url}

%opening
\title{\textbf{Technology Impact Project}}
\author{Adam Sumner}
\date{April 20th, 2015}

\begin{document}

\maketitle

\begin{abstract}
This paper addresses the impact that IIT Alumni Marvin Camras and Martin Cooper have had on the technology industry. The major achievements of each will be recognized, and their impact on society will be compared to each other. The social, economic, ethical, and environmental impacts of their innovations will be addressed, as well as how their innovations affect current and future technological trends. Last, concerns on rapid technology changes will be discussed.
\end{abstract}

\section{Introduction}
IIT and its alumni have had a great impact on society, especially in the world of technological advances. Two of the most notable alumni from the ECE department are Marvin Camras and Martin Cooper. Through their innovations, they paved the way for modern day technology that we use everyday. Camras paved the way for technological advances in the field of magnetic recording, while Cooper revolutionized the wireless communication industry. It is because of them that we are able to use seemingly simple technology such as smart phones and think nothing of it.

\section{Discussion and Answers to Questions}
\subsection{The Major Achievements of Marvin Camras}
Mavin Camras was an electrical engineer and an inventor who was a major influence in the field of magnetic recording. His first device that he built was a simple recording device. This wire recorder was a simple type of storage device that used steel/stainless steel wire to store analog audio\cite{wire}. After this he discovered that using magnetic tape makes the process of splicing audio samples much easier. In June of 1944, he was awarded his first patent, ``US Patent Number 2351004", and later he was awarded over 500 more \cite{patent}. On top of this, Camras wrote a predictive paper titled ``Magnetic Recording and Reproduction - 2012 A.D" which accurately predicted the existence of portable media players the size of a pack of cards without any mechanically moving parts. It is clear now, that his predictions were correct.
\subsection{The Major Achievements of Martin Cooper}
Martin Cooper, like Marvin Camras, was also an electrical engineer and inventor. While working at Motorola, he developed many critical technological advancements that are still in use today, although much more advanced. In 1967, he created the first cellular-like portable handheld police radio system\cite{coop}, and in the early 70's, Cooper became head of Motorola's communications division, where he started to work on his idea of bringing portable cellular phones to the public market. A technology at that time that allowed people to communicate ``portably" was called a car phone, however, they were in limited use. Furthermore, they were specifically designed for use in an automobile, which placed restrictions on its use. On October 17, 1973, Cooper filed his patent for the ``radio telephone system" and was later issued US Patent Number 3,906,166\cite{radio}. Earlier, Cooper and John Francis Mitchell demonstrated two working phones to the media during a scheduled press conference at the New York Hilton in midtown Manhattan. He demonstrated the first cellular phone prototype called the DynaTac\cite{wiki:coop}. This phone only had $\approx$ 20 minutes of talk time before needing a 10 hour battery recharge. This prototype is the precursor to all modern day phone technology that we use today.
\subsection{Global Technological Impacts}
It is clear that these men impacted local society in the United States in a major way. The question, however, is how did this technology impact society globally? Because Camras had developed his technology in the 40's, World War II was greatly affected by his inventions. His wire recorders were not only used to train pilots in the air force, but they were also used for disinformation purposes, such as misdirecting an enemy with amplified battle sounds. This showed the world a new type of battle tactic in which sound can be used to ``flank" the enemy. Cooper's impact globally can be easily seen in that he essentially eliminated boundaries in global communication. No matter where a person is located, if equipped with his technology, they are able to talk to any other person in the world. No longer did the world need to wait for people to be home to communicate. He is responsible for the fact that it is possible to get in contact with anyone at anytime of the day in any location.
\subsection{Social, Economical, Ethical, and Environmental Impacts}
Both technologies affected social life in an extreme way. Cooper enabled people to contact anyone anywhere, while Camras allowed people to create audio recordings for an unlimited amount of playback. Economically, these technologies opened up new possibilities to market gadgets to the public. These technologies created a new type of object for people to spend their money on, thus creating an entirely new market which in turn promoted economic growth. Ethically, Camras' technology pushed the limits on privacy. With the ability to record any audio sound, it could be done secretly, such is the case in the Watergate Scandal with President Nixon. Without a person's approval, is it ok to record what they say? While his invention brought many positive applications, there will always be people who abuse technology for their own gain. This is where ethical issues relating to these innovative technologies come from. It is never the physical object that creates ethical concerns. It's how people use it them. Environmentally, Cooper had the biggest impact which was in the field of infrastructure. With the release of a cellular phone, to be able to link people around the world, cellular towers had to be built to carry wireless signals.
\subsection{Future Technological Trends}
The innovations that these men gave to the world have affected current technology in that they paved the road to get here, and without them, technology would look a lot different.  The smart phone is the biggest example of how Martin Cooper's invention has affected current technology. The cell phone has evolved from the basic DynaTac into a portable computer capable of almost anything, and it also does phone calls. With such a gap in functionality in such a small time elapsed, the technology in the next 10 years could be unthinkable, just as it was to everyone but Marvin Camras. His predictions about the future technology trends were extremely accurate, but it is hard to say where magnetic recording will go next. While specifics cannot be known, it IS known that the next generation of your favorite gadget will be a lot different.
\subsection{Rapid Technological Changes}
Rapid technological changes, while sounding like a good thing, can actually be detrimental towards the progression of technology as a whole. When someone constantly pushes for a better version of a piece of technology and only focuses on that one piece, there's a risk of acquiring a lack of breadth knowledge about current technology trends. Alternatively, if technology trends are allowed to slowly form, all possibilities at each stage of progression can be explored, possibly opening up entirely new branches of tech trends for the future. Personally, I think that by pushing for rapid changes, you approach a wall that can only be passed when other trends catch up to your progression. Why hit the wall sooner and be stuck, when you can progress with the rest of society who needs time to adjust to these new trends.
\section{Conclusion}
Overall, Marvin Camras and Martin Cooper are two of the most notable and successful alumni from the ECE department at IIT. Camras invented magnetic recording, while Cooper invented the first cellular phone. There innovations in technology have positively affected the world as we know it today.
\bibliographystyle{plain}
\bibliography{master}
\end{document}
