\documentclass[12pt]{article}
\usepackage{listings}
\usepackage{color}
\usepackage{float}
\usepackage{graphicx}

\definecolor{mygreen}{rgb}{0,0.6,0}
\definecolor{mygray}{rgb}{0.5,0.5,0.5}
\definecolor{mymauve}{rgb}{0.58,0,0.82}

\lstset{ %
	xleftmargin=2em,
	backgroundcolor=\color{white},   % choose the background color; you must add \usepackage{color} or \usepackage{xcolor}
	basicstyle=\small,%\footnotesize,        % the size of the fonts that are used for the code
	breakatwhitespace=false,         % sets if automatic breaks should only happen at whitespace
	breaklines=false,                 % sets automatic line breaking
	captionpos=b,                    % sets the caption-position to bottom
	commentstyle=\color{mygreen},    % comment style
	deletekeywords={...},            % if you want to delete keywords from the given language
	escapeinside={\%*}{*)},          % if you want to add LaTeX within your code
	extendedchars=true,              % lets you use non-ASCII characters; for 8-bits encodings only, does not work with UTF-8
	%	frame=single,                    % adds a frame around the code
	keepspaces=true,                 % keeps spaces in text, useful for keeping indentation of code (possibly needs columns=flexible)
	keywordstyle=\color{blue},       % keyword style
	language=VHDL, % the language of the code
	morekeywords={*,...},            % if you want to add more keywords to the set
	numbers=left,                    % where to put the line-numbers; possible values are (none, left, right)
	numbersep=5pt,                   % how far the line-numbers are from the code
	numberstyle=\small\color{mygray}, % the style that is used for the line-numbers
	rulecolor=\color{black},         % if not set, the frame-color may be changed on line-breaks within not-black text (e.g. comments (green here))
	showspaces=false,                % show spaces everywhere adding particular underscores; it overrides 'showstringspaces'
	showstringspaces=false,          % underline spaces within strings only
	showtabs=false,                  % show tabs within strings adding particular underscores
	stepnumber=1,                    % step between two line-numbers. If it's 1, each line will be numbered
	stringstyle=\color{mymauve},     % string literal style
	tabsize=2                  % sets default tabsize to 2 spac                  % show the filename of files included with \lstinputlisting; also try caption instead of title
}
%opening
\title{\textbf{Project 3: Designing a 32-bit CPU}}
\author{\textbf{Adam Sumner} - A20283081, \textbf{Contribution} - 25\% \\
		\textbf{Bobby Unverzagt} - A2028923, \textbf{Contribution} - 25\% \\
		\textbf{Emilie Woog} - A20265269, \textbf{Contribution} - 25\% \\
		\textbf{Nash Kaminski} - A20283999, \textbf{Contribution} - 25\% \\ ~\\ ECE 485}
\date{December 5\textsuperscript{th}, 2015}

\begin{document}

\maketitle

\section{Introduction}
This goal of this project is to design a stripped down version of the MIPS processor. The processor will be a 32-bit version of the processor discussed in class and the text book, however, its instruction set will be a small subset of the MIPS processor's full capability. 

\subsection{Background Information}
MIPS is a reduced instruction set computer (RISC) instruction set architecture (ISA). It defines three types of instruction types: R (register), I (Immediate), and J (Jump). For the implementation that this project is focused on, only R and I instructions will be executed. R type instructions are the most common form of instructions. The format for an r-type instruction is:

\begin{center}

\resizebox{\textwidth}{!}{\begin{tabular}{|c|c|c|c|c|c|}
	\hline
	\texttt{Bits[31:26]} & \texttt{Bits[25:21]} & \texttt{Bits[20:16]} & \texttt{Bits[15:11]} & \texttt{Bits[10:6]} & \texttt{Bits[5:0]} \\ \hline
	opcode & Rs & Rt & Rd & shamt & funct \\ \hline
	
\end{tabular}}
\end{center}

\noindent For this instruction, the opcode field is always $000000_2$, while the function code \texttt{funct} is used to determine which instruction is to be carried out. Rs and Rt are the two registers in which the operation reads and Rd is the destination of the result. Some instructions require a shift amount (\texttt{shamt}), so it is specified explicitly.
\\

\noindent The I type instruction involves an immediate value, so the instruction format must accommodate this. The format of this type of instruction is: \\
\begin{center}

\begin{tabular}{|c|c|c|c|}
	\hline
	\texttt{Bits[31:26]} & \texttt{Bits[25:21]} & \texttt{Bits[20:16]} & \texttt{Bits[15:0]} \\ \hline
	opcode & Rs & Rt & immediate \\ \hline
\end{tabular}

\end{center}

\noindent For this isntruction, the op code field is used to define the specific instruction, Rs is the register in which the operation acts on along with the immediate value as the other operand. Rt is the destination register in which the result is stored.

\section{Design}
\subsection{Instruction Set}
Table \ref{tab:operations} shows the instructions that are implemented in the CPU with the respective OpCode and Function Field for each instruction.
\begin{table}[H]
	\centering
	\begin{tabular}{|c|c|c|c|}
		\hline
		\texttt{OpCode[31:26]} & \texttt{Function Field [5:0]} & Instruction & Operation \\ 
		\hline
		$100011_2$ & \texttt{--} & \texttt{lw} & \texttt{lw \$t3, 200(\$s2)}  \\
		\hline
		$101011_2$ & \texttt{--} & \texttt{sw} & \texttt{sw \$t4, 100(\$t3)} \\
		\hline
		$000000_2$ & $100000_2$ & \texttt{add} & \texttt{add \$s3, \$t2, \$s2} \\
		\hline
		$000000_2$ & $110000_2$ & \texttt{sub}& \texttt{sub \$s3, \$t2, \$s2}\\
		\hline
		$000100_2$ & \texttt{--} & \texttt{beq}&\texttt{beq \$s5, \$s2, 500}\\
		\hline
		$000000_2$ & $000001_2$ & \texttt{nand}& \texttt{tbd} \\
		\hline
		$000010_2$ & \texttt{--} & \texttt{andi}& \texttt{tbd} \\
		\hline
		$000000_2$ & $000001_2$ & \texttt{or}& \texttt{tbd} \\
		\hline
		$000011_2$ & \texttt{--} & \texttt{ori}& \texttt{tbd} \\
		\hline		
	\end{tabular}
	\caption{CPU Instruction Set}
	\label{tab:operations}
\end{table}

\subsection{Memory}
For this project, it seemed unnecessary to implement memory of 4GB ($2^{32}$). It was chosen to use an array of 256 words instead. If need be, this memory size could be upgraded easily, so this choice does not hinder performance on the actual design of the CPU.
\subsection{DataPath}

\subsection{Control}

\section{Analysis}

\section{Simulation Results}


\section{Conclusion}
The design and implementation of a 32-bit CPU was a success. An architecture was designed and a behavior was successfully put into practice. All requested functionality was achieved. This 32-bit CPU can now be used in further projects.
\end{document}
