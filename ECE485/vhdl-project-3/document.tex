\documentclass[12pt]{article}
\usepackage{listings}
\usepackage{color}
\usepackage{float}
\usepackage{graphicx}

\definecolor{mygreen}{rgb}{0,0.6,0}
\definecolor{mygray}{rgb}{0.5,0.5,0.5}
\definecolor{mymauve}{rgb}{0.58,0,0.82}

\lstset{ %
	xleftmargin=2em,
	backgroundcolor=\color{white},   % choose the background color; you must add \usepackage{color} or \usepackage{xcolor}
	basicstyle=\small,%\footnotesize,        % the size of the fonts that are used for the code
	breakatwhitespace=false,         % sets if automatic breaks should only happen at whitespace
	breaklines=false,                 % sets automatic line breaking
	captionpos=b,                    % sets the caption-position to bottom
	commentstyle=\color{mygreen},    % comment style
	deletekeywords={...},            % if you want to delete keywords from the given language
	escapeinside={\%*}{*)},          % if you want to add LaTeX within your code
	extendedchars=true,              % lets you use non-ASCII characters; for 8-bits encodings only, does not work with UTF-8
	%	frame=single,                    % adds a frame around the code
	keepspaces=true,                 % keeps spaces in text, useful for keeping indentation of code (possibly needs columns=flexible)
	keywordstyle=\color{blue},       % keyword style
	language=VHDL, % the language of the code
	morekeywords={*,...},            % if you want to add more keywords to the set
	numbers=left,                    % where to put the line-numbers; possible values are (none, left, right)
	numbersep=5pt,                   % how far the line-numbers are from the code
	numberstyle=\small\color{mygray}, % the style that is used for the line-numbers
	rulecolor=\color{black},         % if not set, the frame-color may be changed on line-breaks within not-black text (e.g. comments (green here))
	showspaces=false,                % show spaces everywhere adding particular underscores; it overrides 'showstringspaces'
	showstringspaces=false,          % underline spaces within strings only
	showtabs=false,                  % show tabs within strings adding particular underscores
	stepnumber=1,                    % step between two line-numbers. If it's 1, each line will be numbered
	stringstyle=\color{mymauve},     % string literal style
	tabsize=2                  % sets default tabsize to 2 spac                  % show the filename of files included with \lstinputlisting; also try caption instead of title
}
%opening
\title{\textbf{Project 3: Designing a 32-bit CPU}}
\author{\textbf{Adam Sumner} - A20283081, \textbf{Contribution} - 25\% \\
		\textbf{Bobby Unverzagt} - A2028923, \textbf{Contribution} - 25\% \\
		\textbf{Emilie Woog} - A20265269, \textbf{Contribution} - 25\% \\
		\textbf{Nash Kaminski} - A20283999, \textbf{Contribution} - 25\% \\ ~\\ ECE 485}
\date{December 5\textsuperscript{th}, 2015}

\begin{document}

\maketitle

\section{Introduction}
This goal of this project is to design a stripped down version of the MIPS processor. The processor will be a 32-bit version of the processor discussed in class and the text book, however, its instruction set will be a small subset of the MIPS processor's full capability. 


\begin{table}[H]
	\centering
	\begin{tabular}{|c|c|c|}
		\hline
		\texttt{OpCode[31:26]} & \texttt{Function Field [5:0]} & Instruction \\ 
		\hline
		$100011_2$ & -- & \texttt{lw} \\
		\hline
		$101011_2$ & -- & \texttt{sw}\\
		\hline
		$000000_2$ & $100000_2$ & \texttt{add} \\
		\hline
		$000000_2$ & $110000_2$ & \texttt{sub}\\
		\hline
		$000100_2$ & -- & \texttt{beq}\\
		\hline
		$000001_2$ & $000001_2$ & \texttt{nand} \\
		\hline
		$000010_2$ & $000010_2$ & \texttt{andi} \\
		\hline
		$000011_2$ & $000001_2$ & \texttt{or} \\
		\hline
		$000011_2$ & $000010_2$ & \texttt{ori} \\
		\hline		
	\end{tabular}
	\caption{CPU Instruction Set}
	\label{tab:operations}
\end{table}
\section{Design}
\subsection{Architecture}




\subsection{Behavior}







\section{Analysis}

\section{Simulation Results}


\section{Conclusion}
The design and implementation of an 32-bit CPU was a success. An architecture was designed and a behavior was successfully put into practice. All requested functionality was achieved. This 32-bit CPU can now be used in further projects.
\end{document}
